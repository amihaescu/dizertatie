\documentclass[12pt, a4paper, oneside, romanian]{teza-upb}
\setcounter{secnumdepth}{3}
\setcounter{tocdepth}{3}
\usepackage{babel}
\usepackage{graphicx}
\usepackage[
  bookmarksnumbered,
  bookmarks,
  bookmarksopen=true,
  pdftitle={Dizertatie},
  linktocpage]{hyperref}
\singlespacing
\begin{document}
\author{Andrei Mihaescu}

\title{Arhitecturi software orientate pe servicii}


\facultatea{Facultatea de Electronică, Telecomunicații și Tehnologia Informației}
\tiplucrare{dizertație}
\domeniu{Electronică, Telecomunicații și Tehnologia Informației}
\catedra{Telecomunicații}
\campus{Leu} 
\program{Tehnologii Software Avansate pentru Comunicatii}
\titlulobtinut{Master}
\director{Eduard Popivici} 

\submissionmonth{Iunie} 
\submissionyear{2016} 

\beforepreface
\listoffigures
\listoftables

%\preface{}
\afterpreface 

\chapter{Introducere}
\section{Ce este o arhitectura software?}
Arhitectura software reprezintă procesul de definire a unei soluții structurate care îndeplinește toate cerințele tehnice și operaționale, totodata optimizând metrici comune de calitate precum performanța, securitatea si gestionabilitatea. Aceasta presupune o serie de decizii bazate pe o gamă largă de factori fiecare din aceștia având un impact considerabil asupra calității, performanței, gestionabilitații și bunei funcționarii a aplicației.

\end{document}