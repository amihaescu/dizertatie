\documentclass[12pt, a4paper, oneside, romanian]{teza-upb}
\setcounter{secnumdepth}{3}
\setcounter{tocdepth}{3}
\usepackage{babel}
\usepackage{graphicx}
\usepackage[
  bookmarksnumbered,
  bookmarks,
  bookmarksopen=true,
  pdftitle={Dizertatie},
  linktocpage]{hyperref}
\singlespacing
\begin{document}
\author{Andrei Mihaescu}

\title{Arhitecturi software orientate pe servicii}


\facultatea{Facultatea de Electronică, Telecomunicații și Tehnologia Informației}
\tiplucrare{dizertație}
\domeniu{Electronică, Telecomunicații și Tehnologia Informației}
\catedra{Telecomunicații}
\campus{Leu} 
\program{Tehnologii Software Avansate pentru Comunicatii}
\titlulobtinut{Master}
\director{Eduard Popivici} 

\submissionmonth{Iunie} 
\submissionyear{2016} 

\beforepreface
\listoffigures
\listoftables

%\preface{}
\afterpreface 

\chapter{Introducere}
\section{Ce este o arhitectura software?}
Arhitectura software reprezintă procesul de definire a unei soluții structurate care îndeplinește toate cerințele tehnice și operaționale, totodata optimizând metrici comune de calitate precum performanța, securitatea si facila gestiune. Aceasta presupune o serie de decizii bazate pe o gamă largă de factori fiecare din aceștia având un impact considerabil asupra calității, performanței, gestionabilitații și bunei funcționarii a aplicației.

Philippe Kruchten, Grady Booch, Kurt Bittner și Rich Reitman au derivat și rafinat definiția arhitecturii software bazându-se pe munca lui Mark Shaw si David Garlan (Shawn and Garlan 1996). Definiția lor este următoarea:

"Arhitectura software înglobează setul deciziilor semnificative legate de organizarea unui sistem software ce includ selectarea elementelor structurale și a interfețelor din care sistemul este compus; comportamentul așa cum reiese din interacțiunea acestor elemente; compunerea acestor elemente structurale și comportamentale în subsisteme mai mari; și un stil arhitectural care guvernează această organizare. Arhitectura software implică constrângeri și compromisuri legate de funcționalitate, utilitate, robustețe, performanță, reutilizare, inteligibilitate, economie, tehnice și estetice."

În cartea \emph{"Patterns of Enterprise Application Architecture"}, Martin Fowler evidențiază câteva teme recurente explicând conceptul de arhitectură. El identifică aceste teme dupa cum urmează: "Descompunerea de nivel înalt a unui sistem în parți componente; deciziile care sunt dificil de schimbat; există multe arhitecturi intr-un sistem; ceea ce este arhitectural important se poate schimba de-a lungul ciclului de viata al sistemului; și, la final, arhitectura se rezumă la lucrurile importante."

În cartea \emph{"Software Architecture in Practice (2nd edition)"} Bass, Clements, and Kazman definesc arhitectura astfel: "Arhitectura software a unui program sau a unui sistem de calcul reprezintă structura sau structurile, ce înglobează elementele software, proprietățile lor vizibile către exterior și relația între acestea. Arhitectura se preocupă cu partea public a interfețelor; detaliile private ale elementelor - cele ce sunt strict legate de implementarea internă - nu sunt legate de arhitectură."

\newpage
\section{De ce este arhitectura importantă ?}
Ca orice structură complexă, software-ul trebuie construit pe o bază solidă. Neluarea în considerare a anumitor scenarii cheie, cât și ignorarea anumitor probleme comune de design pot pune aplicația în pericol. Uneltele și platformele moderne ajută la construirea aplicațiilor, însă nu pot înlocui nevoia unuei proiectării atente a aplicației, bazată pe scenarii și cerințe. Riscurile pe care le presupune o arhitectură slab gândită sunt instabilitatea, inabilitatea de a susține cerințele actuale și viitoare de business sau dificultatea de instalare și gestiune într-un mediu de producție.

Un sistem ar trebui proiectat luând în considerare utilizatorul, infrastructura și obiectivele de business. Pentru fiecare din aceste arii, trebuie gândite scenarii cheie, identificate proprietațile relevante și arii cheie de satisfacție și desatisfacție. Unde este posibil, este indicat sa se stabilească metrici precise care vor putea fii evaluate pentru a măsura succesul fiecărei arii.

Cel mai probabil vor exista compromisuri și un echilibru trebuie găsit între cerințele concurente din aceste trei arii. De exemplu experiența utilizatorului, este adesea o funcție de business și infrastructură și schimbările într-una din zone o poate drastic afecta. În mod similar, schimbările la nivelul experienței de utilizare pot avea impact la nivelul infrastructurii și business. Performanța ar putea fi importantă din punct de vedere business și al utilizatorului, dar administratorul de sistem poate nu avea mijloacele financiare pentru a atinge obiectivele în 100\% din timp. Un compromis ar fi să atingă obiectivele 80\% din timp.

Rolul arhitecturii este acela de a defini modul în care elemente majore și componente din cadrul aplicației sunt folosite sau interacționează între ele. Selectarea structurilor de date și a algoritmilor, cât și detaliile de implementare specifice fiecărei componente nu sunt de interes pentru proiectare. Problemele de arhitectură și proiectare de multe ori se intercalează. In unele cazuri, deciziile țin mai mult de arhitectură. In altele, în schimb țin mai mult de proiectare și de cum acestea contribuie la realizarea arhitecturii.

\section{Obiectivele arhitecturii software}

Arhitectura software urmărește să găsească un compromis între specificațiile de business și cele tehnice înțelegând cazurile de utilizare și apoi găsind căi de implementare. Obiectivele arhitecturii sunt acelea de a găsi cerințele care influențează structura aplicației. O arhitectură bine realizată reduce riscurile de business asociate cu construirea unei soluții tehnice. Un design bun este suficient de flexibil ca să poate gestiona devierile de la tehnologiile software si hardwre care pot apărea în timp, cât și cerințele utilizatorilor. Un arhitect trebuie să ia in considerare efectul global al deciziilor de proiectare, cât și compromisurile inerente între factorii de perfomanță, dar și cele cu privire la utilizator, infrastructură și cerințele de business.

\newpage
\section{Principii arhitecturale cheie}
Pentru proiectarea unei arhitecturii următoarele principii cheie ar trebui luate în calcul:
\begin{itemize}
	\item \textbf{Arhitectura trebuie sa fie concepută pentru a suporta schimbări, nu pentru a rămâne neschimbătă.} Întotdeauna trebuie luat în calcul cum aplicația s-ar putea modifica de-a lungul timpului pentru a putea răspunde noilor cerințe și provocări.
	\item \textbf{Modelarea trebuie facută pentru reduce riscul.} Este indicată folosirea uneltelor de proiectare, a sistemelor de modelare precum \textbf{UML -} \emph{Unified Modeling Language} și a vizualizare unde este cazul pentru a putea evidenția cerințele și a deciziile arhitecturale și de proiectare, analizându-le impactul.
	\item \textbf{Folosirea modelelor și a uneltelor de vizualizare pentru comunicare si colaborare.} Comunicarea eficientă a designului, a deciziilor luate și schimbărilor necesare este necesară unei arhitecturi bune. Este recomandată folosirea modelelor, a vederilor și a altor mijloace de vizualire a arhitecturii pentru a comunica și impărtăși eficient ideile cu clienții, rezultând astfel într-o comunicare rapidă a schimbărilor de proiectare.
	\item \textbf{Identificarea deciziilor tehnice cheie.} Este esențiala înțelegerea deciziilor tehnice cheie pentru evitarea greșelilor comune. Investirea în luarea acestor decizii bine de prima dată este importantă pentru a obține un design flexibil și foarte puțin expus riscului de a fi afectat de viitoare schimbări.
\end{itemize}
\end{document}