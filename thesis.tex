\documentclass[12pt, a4paper, oneside, romanian]{teza-upb}
\setcounter{secnumdepth}{3}
\setcounter{tocdepth}{3}
\usepackage{babel}
\usepackage{graphicx}
\usepackage[
  bookmarksnumbered,
  bookmarks,
  bookmarksopen=true,
  pdftitle={Dizertatie},
  linktocpage]{hyperref}
\singlespacing
\begin{document}
\author{Andrei Mihaescu}

\title{Arhitecturi software orientate pe servicii}


\facultatea{Facultatea de Electronică, Telecomunicații și Tehnologia Informației}
\tiplucrare{dizertație}
\domeniu{Electronică, Telecomunicații și Tehnologia Informației}
\catedra{Telecomunicații}
\campus{Leu} 
\program{Tehnologii Software Avansate pentru Comunicatii}
\titlulobtinut{Master}
\director{Eduard Popivici} 

\submissionmonth{Iunie} 
\submissionyear{2016} 

\beforepreface
\listoffigures
\listoftables

%\preface{}
\afterpreface 

\chapter{Introducere}
\section{Ce este o arhitectura software?}
Arhitectura software reprezintă procesul de definire a unei soluții structurate care îndeplinește toate cerințele tehnice și operaționale, totodata optimizând metrici comune de calitate precum performanța, securitatea si facila gestiune. Aceasta presupune o serie de decizii bazate pe o gamă largă de factori fiecare din aceștia având un impact considerabil asupra calității, performanței, gestionabilitații și bunei funcționarii a aplicației.

Philippe Kruchten, Grady Booch, Kurt Bittner și Rich Reitman au derivat și rafinat definiția arhitecturii software bazându-se pe munca lui Mark Shaw si David Garlan (Shawn and Garlan 1996). Definiția lor este următoarea:

"Arhitectura software înglobează setul deciziilor semnificative legate de organizarea unui sistem software ce includ selectarea elementelor structurale și a interfețelor din care sistemul este compus; comportamentul așa cum reiese din interacțiunea acestor elemente; compunerea acestor elemente structurale și comportamentale în subsisteme mai mari; și un stil arhitectural care guvernează această organizare. Arhitectura software implică constrângeri și compromisuri legate de funcționalitate, utilitate, robustețe, performanță, reutilizare, inteligibilitate, economie, tehnice și estetice."

În cartea \emph{"Patterns of Enterprise Application Architecture"}, Martin Fowler evidențiază câteva teme recurente explicând conceptul de arhitectură. El identifică aceste teme dupa cum urmează: "Descompunerea de nivel înalt a unui sistem în parți componente; deciziile care sunt dificil de schimbat; există multe arhitecturi intr-un sistem; ceea ce este arhitectural important se poate schimba de-a lungul ciclului de viata al sistemului; și, la final, arhitectura se rezumă la lucrurile importante."

În cartea \emph{"Software Architecture in Practice (2nd edition)"} Bass, Clements, and Kazman definesc arhitectura astfel: "Arhitectura software a unui program sau a unui sistem de calcul reprezintă structura sau structurile, ce înglobează elementele software, proprietățile lor vizibile către exterior și relația între acestea. Arhitectura se preocupă cu partea public a interfețelor; detaliile private ale elementelor - cele ce sunt strict legate de implementarea internă - nu sunt legate de arhitectură."


\end{document}