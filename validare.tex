\documentclass[12pt, a4paper, oneside, romanian]{teza-upb}
\setcounter{secnumdepth}{3}
\setcounter{tocdepth}{3}
\usepackage{babel}
\usepackage{graphicx}
\usepackage{float}
\usepackage[
  bookmarksnumbered,
  bookmarks,
  bookmarksopen=true,
  pdftitle={Dizertatie},
  linktocpage]{hyperref}
\singlespacing
\begin{document}
\begin{titlepage}
	\centering
	\includegraphics[width=0.15\textwidth]{img/UPB-logo.png}\par\vspace{1cm}
	{\scshape\LARGE Facultatea de Electronică, Telecomunicații și Tehnologia Informației \par}
	\vspace{1cm}
	{\huge\bfseries Proiect validare\par}
	\vspace{2cm}
	{\Large\bfseries Diffie Hellman\par}
	\vspace{2cm}
	{\Large\bfseries Andrei Mihăescu\par}
	{\large TSAC II, ianuarie 2016\par}

% Bottom of the page
\end{titlepage}
\newpage

\chapter{Prezentare protocol}	
Metoda schimbului de chei Diffie-Hellman, cunoscută şi ca metoda de distribuţie a cheilor publice, poartă numele a doi specialişti de la Standford University, Whitfield Diffie şi Martin Hellman. În anul 1976, ei au inventat o metodă prin care două părţi pot cădea de comun acord să comunice prin mesaje secrete fără să fie nevoie de o terţă parte, de un schimb off-line sau de transmiterea vreunei valori secrete între ele.

Independent, Ralph Merkle a venit cu o soluţie de distribuţie a cheilor publice, numai că metoda propusă implica substanţiale cheltuieli pentru efectuarea calculelor şi a transmisiei. Varianta realizată de Diffie şi Hellman a fost numită sistemul distribuţiei cheilor publice sau al schimburilor de chei publice.

Metoda Diffie-Hellman se bazează pe conceptul perechii de chei publică privată.
Protocolul începe cu fiecare parte care generează independent câte o cheie privată. În pasul următor, fiecare calculează câte o cheie publică, aceasta fiind o funcţie matematică a cheilor private respective. Urmează schimbul de chei publice.

În final, fiecare dintre cele două persoane calculează o funcţie a propriei chei private şi a cheii publice a celeilalte persoane. Matematica este cea care va face să se ajungă la aceeaşi valoare, care este derivată din cheile lor private. Ele vor folosi valoarea ca pe cheie a mesajului.

Diffie şi Hellman folosesc exponenţierea în aritmetica modulară pentru a calcula cheile publice şi cheia mesajului. Aritmetica modulară este ca şi aritmetica standard, cu excepţia faptului că foloseşte numere numai în intervalul 0 la N, numit modulo. Atunci când o operaţie produce un rezultat care este mai mare sau egal cu N, N este scăzut repetat din rezultat până când valoarea se încadrează în intervalul 0 la N-1 (ca şi cum s-ar împărţi la N şi se ia în seamă restul). De exemplu, 3+4 mod 5 = 2. Dacă rezultatul este negativ, N se adaugă acestuia până când se va încadra în intervalul 0 la N-1. De exemplu, 3-8 mod 7 = -5 mod 7 = 2.

În aritmetica modulară, exponenţierea este o funcţie într-un singur sens.
Aceasta înseamnă că este uşor de calculat un număr y = gx mod N pentru o valoare secretă x, însă este mult mai dificil să se calculeze x din y, dacă numerele sunt suficient de mari, ca de exemplu o lungime de câteva sute de cifre (noi presupunem că g şi N sunt cunoscute). Aceasta este referită ca şi problema logaritmului discret pentru că x este logaritm din y în baza g (mod N), iar numerele sunt finite şi întregi.
\begin{figure}[tb]
\centering
\includegraphics*[scale=0.3]{img/dhke.png}
\caption{Functionarea protocolului Diffie Hellman}
\label{fig:punguta}
\end{figure}
Cu metoda Diffie-Hellman a schimbului de chei publice, Alice şi Bob stabilesc cheia mesajului secret după cum urmează. Alice generează o cheie secretă xa şi Bob o cheie secretă xb. După aceasta, Alice calculează o cheie publică ya, care este g ridicat la puterea xa modulo p, unde p este un număr prim (adică nu poate fi descompus în produsul a două numere), g fiind mai mic decât p. Identic, Bob calculează o cheie publică yb, prin ridicarea lui g la puterea xb modulo p. Ei vor schimba valorile publice ale acestora. Apoi, Alice ridică cheia publică a lui Bob la puterea exponentului său, xa modulo p, în timp ce Bob ridică cheia publică a lui Alice la exponentul său, xb modulo p. Amândoi vor obţine acelaşi rezultat, g ridicat la puterea xa şi xb, iar rezultatul obţinut va fi folosit de amândoi drept cheia K a mesajului. Matematic, totul se va exprima astfel:




Deşi în practică se folosesc numere foarte lungi, de câteva sute de cifre, pentru a ajuta la înţelegerea modului de funcţionare, vom folosi numere mici.
\vspace{1cm}

Exemplul 1

Să presupunem că p = 7, g = 3, cheia lui Alice xa = 1 şi a lui Bob xb = 2
Vom avea:
\begin{itemize}
 \item Alice calculează cheia sa publică: ya = gxa mod p = 31 mod 7 = 3
 \item Bob calculează cheia sa publică: yb = gxb mod p = 32 mod 7 = 2
 \item Alice calculează K = ybxa mod p = 21 mod 7 = 2
 \item Bob calculează K = yaxb mod p = 32 mod 7 = 2
\end{itemize}
sau
K = gxa*xb mod p = 32×1 mod 7 = 9 mod 7 = 2.
\vspace{1cm}

Exemplul 2

Să presupunem că p = 5, g = 4, cheia lui Alice xa = 3 şi a lui Bob xb = 2
\begin{itemize}
 \item Alice calculează cheia sa publică: ya = gxa mod p = 43 mod 5 = 4
 \item Bob calculează cheia sa publică: yb = gxb mod p = 42 mod 5 = 1
 \item Alice calculează K = ybxa mod p = 13 mod 5 = 1
 \item Bob calculează K = yaxb mod p = 42 mod 5 = 1
\end{itemize}
sau
K = gxa*xb mod p = 43×2 mod 5 = 4096 mod 5 = 1.
Se observă că în ambele cazuri K ia valori identice, 2, respectiv 1.

Metoda Diffie-Hellamn, precum şi variantele ei sunt utilizate în câteva protocoale de securitate a reţelelor şi în produse comerciale, inclusiv la AT\&T 3600 Telephone Security Device, la Fortezza card – o variantă de carduri criptate, şi la Pretty Good Privacy pentru criptarea e-mail-urilor şi a unor fişiere.
\newpage
\chapter{Atacuri asupra protocolului}
Acest tip de protocol este vulnerabil la atacurile de tip "Man-in-the-middle" care functioneaza dupa principiul ilustrat mai jos.
\begin{figure}[th]
\centering
\includegraphics*[scale=1]{img/dh-mitm.png}
\caption{Atac de tip Man-in-the-Middle asupra protocolului Diffie-Hellman}
\label{fig:punguta}
\end{figure}
\end{document}